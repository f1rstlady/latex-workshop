\documentclass[12pt]{scrreprt}

% Sprache
% Diese Packages sorgen dafür, dass man ganz einfach Deutsch schreiben kann.
\usepackage[ngerman]{babel}

% Schrift: Latin Modern
\usepackage{lmodern}

% Mathe
% Diese Packages sorgen dafür, dass man alle möglichen Matheformeln parat hat.
\usepackage{amsmath}
\usepackage{amssymb}
\usepackage{amsthm}

% Umgebungen
% Beim ersten \newtheorem: {Aufrufbefehl}{Name}[Nummerierung]
\newtheorem{satz}{Satz}[section]
% Beim zweiten \newtheorem: {Aufrufbefehl}[Nummerierung]{Name}
\newtheorem{lemma}[satz]{Lemma}
\newtheorem{korollar}[satz]{Korollar}
\theoremstyle{definition}
\newtheorem{definition}[satz]{Definition}
\newtheorem{beispiel}[satz]{Beispiel}
\theoremstyle{remark}
\newtheorem{bemerkung}[satz]{Bemerkung}

% Verkürzte Befehle
% Man will nicht immer lange Befehle schreiben, daher definiert man sich hier kurze Befehle.
\newcommand{\N}{\mathbb{N}}
\newcommand{\Z}{\mathbb{Z}}
\newcommand{\Q}{\mathbb{Q}}
\newcommand{\R}{\mathbb{R}}
\newcommand{\C}{\mathbb{C}}

\begin{document}

\chapter{Mein erster Code!}

\section{Hinführung}
In diesem Kapitel wollen wir den \textbf{Mittelwertsatz} beweisen.  Dazu benötigen wir zuerst diesen

\begin{lemma}
    Ist die Funktion $f$ differenzierbar, so ist sie auch stetig.
\end{lemma}

Wir formulieren nun den

\begin{satz}[Satz von Rolle]
    Sei $f \colon [a,b] \to \R$ stetig und auf $(a,b)$ differenzierbar, es sei zusätzlich $f(a) =
    f(b)$.  Dann existiert ein $\xi \in (a,b)$ mit $f'(\xi) = 0$.
\end{satz}

Wir verwenden diesen Satz ohne Beweis.

\section{Eine Perle der Analysis}

\begin{satz}[Mittelwertsatz]
    Sei $f \colon [a,b] \to \R$ stetig und auf $(a,b)$ differenzierbar.  Dann existiert ein $\xi \in
    (a,b)$ mit
        \[ f'(\xi) = \frac{f(b)-f(a)}{b-a} \]
\end{satz}
\begin{proof}
    Man betrachte die Funktion $g \colon [a,b] \to \R$ mit
    \[ g(x) = \frac{f(b)-f(a)}{b-a} \cdot (x-a) - f(x) \]
    Es gilt, dass $g(a) = f(a) = g(b)$ und dass $g$ als Komposition differenzierbarer Funktion
    wieder differenzierbar ist.  Aus dem Satz von Rolle folgt nun leicht die Behauptung.
\end{proof}

\section{Ausblick}
Im Mehrdimensionalen definieren wir den Gradienten:
    \[ \nabla f =
        \begin{pmatrix}
            \frac{\partial f}{\partial x_1} \\
            \vdots \\
            \frac{\partial f}{\partial x_n}
        \end{pmatrix}
        \begin{pmatrix}
            x \\
            y
        \end{pmatrix}
    \]

\end{document}
