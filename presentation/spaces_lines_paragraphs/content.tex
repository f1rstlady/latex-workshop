\subsection{Leerzeichen, Zeilenumbrüche \& Paragraphen}

\begin{frame}[fragile]{Leerzeichen}
    \begin{itemize}
        \item Wörter werden durch Leerzeichen getrennt.  Die Anzahl der Leerzeichen ist aber
            irrelevant:
            \example*{examples/whitespace.tex}
        \item Möchte man unbedingt ein Leerzeichen erzeugen, so muss man eine Tilde einfügen:
            \example*{examples/whitespace_non-breaking.tex}
        \item Dies ist in der Regel nicht notwendig, da \LaTeX~normalerweise automatisch den
            richtigen Abstand zwischen Wörten bestimmt.
    \end{itemize}
\end{frame}

\begin{frame}[fragile]{Zeilenumbrüche}
    \begin{itemize}
        \item Zeilenumbrüche im Quellcode übertragen sich im Allgemeinen nicht auf die PDF:
            \example*{examples/newline.tex}
        \item Man kann durch \code{\\} einen Zeilenumbruch erzwingen:
            \example*{examples/newline_forced.tex}
        \item Auch hier gilt: Passende Stellen für Zeilenumbrüche werden in der Regel automatisch
            bestimmt.
    \end{itemize}
\end{frame}

\begin{frame}[fragile]{Paragraphen}
    \begin{itemize}
        \item Um Text in Paragraphen zu teilen, muss man sie im Quellcode durch eine Leerzeile
            trennen:
            \example*{examples/paragraph.tex}
        \item Alternativ erzeugt der Befehl \code{\par} einen neuen Paragraphen.
        \item Paragraphen werden standardmäßig nicht durch vertikalen Abstand, sondern Einrückung
            hervorgehoben.
    \end{itemize}
\end{frame}
