\subsection{Sätze, Lemmata und Beweise}

\begin{frame}[fragile]{\subsecname}
    \begin{itemize}
        \item Um elegant und einfach Sätze, Lemmata, Beweise, etc. zu setzen, eignet sich das Paket
            \texttt{amsthm}:
            \begin{center}
                \code{\usepackage{amsthm}}
            \end{center}
        \item Es stellt Makros bereit um entsprechende Umgebungen zu erstellen und benutzen:
            \example{example.tex}
            \begin{itemize}
                \item Da das Erscheinungsbild in dieser Präsentation stark von dem in anderen
                    Dokumentklassen abweicht, probiert es selbst mal aus.
            \end{itemize}
    \end{itemize}
\end{frame}

\begin{frame}[fragile]{\subsecname~(2)}
    \begin{itemize}
        \item Der Befehl \code{\newtheorem{<env_name>}{<name>}} erstellt eine neue
            Theorem\-/Umgebung namens \texttt{<env\_name>}, die im Dokument als \texttt{<name>}
            erscheint.
            \begin{itemize}
                \item Damit lassen sich insbesondere Umgebungen für Lemmata, Beispiele, Bemerkungen
                    etc. erstellen.
            \end{itemize}
        \item Wie für Abschnitte gibt es auch für Theorem\-/Umgebungen unnummmerierte Varianten:
            \begin{center}
                \code{\newtheorem*{<env_name>}{<name>}}
            \end{center}
            funktioniert wie \code{\newtheorem}, erstellt aber eine unnummerierte Umgebung.
    \end{itemize}
\end{frame}
