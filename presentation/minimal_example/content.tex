\subsection{Ein Minimalbeispiel}

\begin{frame}[fragile]{\subsecname}
    \begin{itemize}
        \item Beim \enquote{\TeX{}en} verfasst man zunächst \alert{Quellcode} in einer simplen
            Textdatei mit \texttt{.tex}\-/Endung:
            \example{example.tex}
        \item Ein \alert{Compiler} erzeugt daraus ein gewünschtes Output\-/Format:
            \begin{center}
                \texttt{> lualatex HelloWorld.tex}
            \end{center}
        \item Es gibt verschiedene Compiler (für PDF\-/Ausgabe):
            \begin{itemize}
                \item \texttt{pdflatex} wurde und wird noch immer viel verwendet.
                \item \texttt{lualatex} ist moderner und gilt als dessen Nachfolger.
            \end{itemize}
            Der Compiler muss ggf. im Editor eingestellt werden.
    \end{itemize}
\end{frame}
