\subsection{Listen, Aufzählungen, Beschreibungen}

\begin{frame}[fragile]{Listen}
    \begin{itemize}
        \item Ein weiteres nützliches Hilfsmittel sind \alert{Listen}. Ein hilfreiches Paket dafür
            ist \texttt{enumitem}, das wie folgt in der Präambel eingebunden werden sollte:
            \begin{center}
                \code{\usepackage[shortlabels]{enumitem}}
            \end{center}
        \item Eine (nicht\-/nummerierte) Liste erzeugt man mit der Umgebung \texttt{itemize}:
            \example*{examples/itemize.tex}
        \item Das Aufzählungssymbol kann lokal durch eckige Klammern (\code{\item[...]}) oder
            global in der Präambel (\code{\setlist[itemize]{label=...}}) gesetzt werden.
    \end{itemize}
\end{frame}

\begin{frame}[fragile]{Aufzählungen}
    \begin{itemize}
        \item Nummerierte Listen lassen sich mit der Umgebung \texttt{enumerate} erzeugen:
            \example*{examples/enumerate.tex}
        \item Durch die eckigen Klammern nach \code{\begin{enumerate}} kann die Nummerierung
            und Klammerung angepasst werden.
            \begin{itemize}
                \item z.B. \code{[(i)], [1)], [A.],} \dots
            \end{itemize}
    \end{itemize}
\end{frame}

\begin{frame}[fragile]{Beschreibungen}
    \begin{itemize}
        \item Die Umgebung \texttt{description} eignet sich zur Auflistung von Begriffen mit
            zugehörigen Beschreibungen:
            \example*{examples/description.tex}
    \end{itemize}
\end{frame}
