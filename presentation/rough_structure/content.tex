\subsection{Grober Aufbau}

\begin{frame}{\subsecname}
    \begin{itemize}
        \item Eine \TeX\-/Datei (\texttt{.tex}) beginnt in der Regel mit einer sog.
            \alert{Präambel}:
            \example{examples/preamble.tex}
        \item In dieser werden alle Einstellungen des Dokuments festgelegt, wie z.B. Layout, eigene
            Befehle und zusätzliche Pakete. Dies erzeugt noch keinen sichtbaren Output!
        \item Notwendig: \code{\\documentclass\{...\}} zu Beginn zum Festlegen der
            \alert{Dokumentklasse}.
            \begin{itemize}
                \item Empfohlen: KOMA\-/Script\-/Klassen (\texttt{scrartcl}, \texttt{scrreprt},
                    \texttt{scrbook}, \dots) statt der Standardklassen (\texttt{article},
                    \texttt{report}, \texttt{book}, \dots)
            \end{itemize}
    \end{itemize}
\end{frame}

\begin{frame}{\subsecname~(2)}
    \begin{itemize}
        \item Der eigentliche Inhalt, der ausgegeben werden soll, befindet sich in der Umgebung
            \texttt{document}:
            \example{examples/document.tex}
        \item Text nach \code{\\end\{document\}} wird nicht verarbeitet.
    \end{itemize}
\end{frame}
