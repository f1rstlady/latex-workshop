\documentclass[german]{f1rstlady/templates/presentation}

% Source code integration
\usepackage{f1rstlady/sourcecode}
\lstset{
    language = [LaTeX]{TeX},
    inputpath = examples,
    keywordstyle = \color{Green4},
}
\lstdefinestyle{custom}{
    style = context,
    basicstyle = \ttfamily\scriptsize,
    caption = {},
}
\newcommand{\monospaced}[1]{\texttt{\small #1}}

% Macro to display examples
\newcommand{\code}[1]{
    \begin{description}
        \item[Input:]
            \lstinputlisting{#1}
        \item[Output:]
            \input{examples/#1}
    \end{description}
}

% TeX workflow diagram
\usepackage{smartdiagram}

% Information to be included in the title page:
\title{\LaTeX\ Workshop}
\author{Julian Kranz, Benedikt Rips}
\institute[]{Learning Center}
\date{\today}

\begin{document}

\frame[plain]{\titlepage}

\frame{\frametitle{Übersicht}\tableofcontents}

\AtBeginSection[]
{
    \begin{frame}
        \tableofcontents[currentsection, sectionstyle=show/shaded]
    \end{frame}
}

\begin{frame}{Was ist \LaTeX?}
\begin{itemize}
    \item Ein Textsatzsystem zum Setzen ansprechender Texte mit mathematischen Inhalten.
        (Eigentlich: Erweiterung für das Textsatzsystem TeX)
    \item De-facto-Standard in Mathematik, Informatik, Naturwissenschaften und mathematiknahen
        Disziplinen
    \item Kostenlose Open Source Software
\end{itemize}
\end{frame}

\begin{frame}{Warum \LaTeX?}
\LaTeX\ bietet zahlreiche Vorteile gegenüber gängigen Textverarbeitungsprogrammen:
\begin{itemize}
    \item Klare Trennung von Inhalt und Formatierung
    \item Einfacheres und mächtigeres Setzen mathematischer Formeln
    \item Plattformunabhängig
    \item Flexibilität – diese Präsentation wurde mit LaTeX erstellt!
    \item Ausgeprägte Modularität
    \item Automatisiertes Erstellen von Abschnittsnummerierungen, Inhalts- und
        Literaturverzeichnissen, \dots
    \item Zuverlässiges Zitieren und cross referencing
    \item Programmierbar durch Kontrollstrukturen
\end{itemize}
\end{frame}

\section{Einführung}

\subsection{Ein Minimalbeispiel}

\begin{frame}{\subsecname}
\begin{itemize}
    \item Beim \enquote{TeXen} verfasst man zunächst \alert{Quellcode} in einer simplen Textdatei
        mit \monospaced{.tex}-Endung:
        \lstinputlisting[style=custom]{minimal.tex}
    \item Ein \alert{Compiler} erzeugt daraus ein gewünschtes Output-Format:
        \begin{center}
            \monospaced{> lualatex HelloWorld.tex}
        \end{center}
    \item Es gibt verschiedene Compiler (für PDF-Ausgabe):
        \begin{itemize}
            \item \alert{\monospaced{pdflatex}} wurde und wird immer noch viel verwendet.
            \item \alert{\monospaced{lualatex}} ist modern und gilt als dessen Nachfolger.
        \end{itemize}
        Der Compiler muss ggf. im Editor eingestellt werden.
\end{itemize}
\end{frame}

\subsection{Grober Aufbau}

\begin{frame}{\subsecname}
\begin{itemize}
    \item Eine TeX-Datei (\monospaced{.tex}) beginnt in der Regel mit einer sog. \alert{Präambel}:
        \lstinputlisting[style=custom]{preamble.tex}
    \item In dieser werden alle Einstellungen des Dokuments festgelegt, wie z.B. Layout, eigene
        Befehle und zusätzliche Pakete. \alert{Dies erzeugt noch keinen sichtbaren Output!}
    \item Notwendig: \lstinline{\\documentclass\{...\}}-Befehl zu Beginn zum Festlegen der
        Dokumentklasse.
        \begin{itemize}
            \item Empfohlen: \alert{KOMA-Script-Klassen} (\monospaced{scrartcl},
                \monospaced{scrreprt}, \monospaced{scrbook}, \dots) statt der Standardklassen
                (\monospaced{article}, \monospaced{report}, \monospaced{book}, \dots)
        \end{itemize}
\end{itemize}
\end{frame}

\begin{frame}{Grober Aufbau (2)}
\begin{itemize}
    \item Der eigentliche Inhalt, der ausgegeben werden soll, befindet sich in der
        \monospaced{document}-Umgebung:
        \lstinputlisting[style=custom]{document.tex}
    \item Text nach \lstinline{\\end\{document\}} wird nicht verarbeitet.
\end{itemize}
\end{frame}

% TODO: Whitespace behandeln
%\subsection{Whitespace}

\subsection{Kommentare}

\begin{frame}{\subsecname}
\begin{itemize}
    \item Alles, was einem \enquote{\monospaced{\%}} folgt, wird ignoriert und taucht nicht im
        generierten Dokument auf:
        \code{comments.tex}
    \item Dies erlaubt es, den Text mit \alert{Kommentaren} zu versehen.
    \item Nützlich, um Gedanken und Erklärungen und Notizen aufzuschreiben, die vor dem Leser
        verborgen bleiben sollen.
\end{itemize}
\end{frame}

\begin{frame}{TeX-Workflow}
\begin{center}
    \smartdiagram[circular diagram:clockwise]{Quellcode schreiben, kompilieren, Output begutachten}
\end{center}
\end{frame}

\subsection{Abschnitte und Überschriften}

\begin{frame}{\subsecname}
\begin{itemize}
    \item Man kann ein Dokument in Abschnitte unterteilen mittels
        \begin{center}
            \lstinline{\\section\{Überschrift\}}
        \end{center}
    \item Neben \lstinline{\\section} gibt es z.B. \lstinline{\\chapter}, \lstinline{\\subsection}
        und \lstinline{\\paragraph}.
    \item Die Abschnitte werden automatisch nummeriert.
    \item Zu jedem dieser Makros gibt es eine unnummmerierte Variante, die durch einen Stern
        gekennzeichnet ist:
        \begin{center}
            \lstinline{\\section*} statt \lstinline{\\section}
        \end{center}
    \item Mittels \lstinline{\\tableofcontents} erzeugt man aus den Abschnitten ein
        Inhaltsverzeichnis.
\end{itemize}
\end{frame}

\subsection{Listen, Aufzählungen, Beschreibungen}

\begin{frame}{Listen}
\begin{itemize}
    \item Ein weiteres nützliches Hilfsmittel sind \alert{Listen}. Ein hilfreiches Paket dafür ist
        \monospaced{enumitem}, das wie folgt in der Präambel eingebunden werden sollte:
        \begin{center}
            \lstinline{\\usepackage[shortlabels]\{enumitem\}}
        \end{center}
    \item Eine (nicht-nummerierte) Liste erzeugt man mit der \monospaced{itemize}-Umgebung:
        \code{itemize.tex}
    \item Aufzählungssymbol kann lokal durch eckige Klammern (\lstinline{\\item[...]}) oder global
        in der Präambel (\lstinline{\\setlist[itemize]\{label=...\}}) gesetzt werden.
\end{itemize}
\end{frame}

\begin{frame}{Aufzählungen}
\begin{itemize}
    \item Nummerierte Listen lassen sich mit der \monospaced{enumerate}-Umgebung erzeugen:
        \code{enumerate.tex}
    \item Durch die eckigen Klammern nach \lstinline{\\begin{enumerate}} kann die Nummerierung und
        Klammerung angepasst werden.
        \begin{itemize}
            \item z.B. \lstinline{[(i)], [1)], [A.],} \dots
        \end{itemize}
\end{itemize}
\end{frame}

\begin{frame}{Beschreibungen}
\begin{itemize}
    \item Die \monospaced{description}-Umgebung eignet sich zur Auflistung von Begriffen mit
        zugehörigen Beschreibungen:
        \code{description.tex}
\end{itemize}
\end{frame}

\subsection{Mathematische Formeln}

\begin{frame}{\subsecname}
Mathematische Formeln können auf verschiedene Arten und Weisen im so genannten \alert{math mode}
gesetzt werden:
\begin{itemize}
    \item Innerhalb von Fließtext (\alert{inline math}) mittels \lstinline{$...$}:
        \code{inline_math.tex}
    \item Abgesetzt in einer separaten Zeile (\alert{display math}) mittels \lstinline{[...]}:
        \code{display_math.tex}
\end{itemize}
\end{frame}

\begin{frame}{Indizes und Exponenten}
\begin{itemize}
    \item Indizes:
        \code{indices.tex}
        Beachte: Für Indizes mit mehr als einem Zeichen muss der Index in geschweifte Klammern
        gesetzt werden: \lstinline{v_\{i-1\}}.
    \item Exponenten:
        \code{exponents.tex}
        Beachte: Wie bei Indizes müssen mehrere Zeichen in geschweifte Klammern gesetzt werden:
        \lstinline{n^\{k+1\}}.
\end{itemize}
\end{frame}

\begin{frame}{Mehrzeilige Formeln}
\begin{itemize}
    \item Für mehrzeilige Rechnungen bieten sich Umgebungen wie \monospaced{align} an:
        \code{align.tex}
    \item Die Zeilen werden dabei anhand der \lstinline{&} ausgerichtet.
    \item Andere nützliche Umgebungen sind \monospaced{equation}, \monospaced{gather},
        \monospaced{array}, \monospaced{multiline}.
    \item Diese Umgebungen werden vom Paket \monospaced{amsmath} bereitgestellt, welches daher bei
        Bedarf in der Präambel eingebunden werden muss:
        \begin{center}
            \lstinline{\\usepackage\{amsmath\}}
        \end{center}
    \item Nummerierung kann mit \lstinline{\\nonumber} pro Zeile ausgeschaltet werden – oder man
        nutzt \monospaced{align*} statt \monospaced{align}.
\end{itemize}
\end{frame}

\begin{frame}[t]{Mathematische Symbole}
\begin{itemize}
    \item Das Paket \monospaced{amssymb} enthält weitere mathematische Symbole:
        \begin{center}
            \lstinline{\\usepackage\{amssymb\}}
        \end{center}
    \item Falls man einen Befehl für ein bestimmtes mathematisches Symbol sucht, hilft der Dienst
        \href{http://detexify.kirelabs.org/classify.html}{\alert{Detexify}} weiter:
        \begin{center}
            \includegraphics[keepaspectratio,width=8cm]{images/detexify.png}
        \end{center}
\end{itemize}
\end{frame}

\begin{frame}{Sätze, Lemmata und Beweise}
\begin{itemize}
    \item Um elegant und einfach Sätze, Lemmata, Beweise, etc. zu setzen, eignet sich das Paket
        \monospaced{amsthm}:
        \begin{center}
            \lstinline{\\usepackage\{amsthm\}}
        \end{center}
    \item Es stellt Makros bereit um entsprechende Umgebungen zu erstellen und benutzen:
        \lstinputlisting[style=custom]{theorem.tex}
    \item Da das Erscheinungsbild in dieser Präsentation stark von dem in anderen Dokumentklassen
        abweicht, probiert selbst mal aus.
\end{itemize}
\end{frame}

\begin{frame}{Sätze, Lemmata und Beweise (2)}
\begin{itemize}
    \item Der Befehl \lstinline{\\newtheorem\{<env_name>\}\{<name>\}} erstellt eine neue
        Theorem-Umgebung namens \monospaced{<env\_name>}, die im Dokument als \monospaced{<name>}
        erscheint.
        \begin{itemize}
            \item Damit lassen sich insbesondere Umgebungen für Lemmata, Beispiele, Bemerkungen etc.
                erstellen.
        \end{itemize}
    \item Wie für Abschnitte gibt es auch für Theorem-Umgebungen unnummmerierte Varianten:
        \begin{center}
            \lstinline{\\newtheorem*\{<env_name>\}\{<name>\}}
        \end{center}
        funktioniert wie \lstinline{\\newtheorem}, erstellt aber eine unnummerierte Umgebung.
\end{itemize}
\end{frame}

\section{Übung}

\begin{frame}{Aufgabe}
\begin{itemize}
    \item Im folgenden haben wir für euch ein beispielhaftes Dokument erstellt, dieses findet sich
        im Learnweb.
    \item Zu Übungszwecken empfehlen wir euch, einmal selbst zu versuchen das Dokument zu TeXen.
    \item Bei Fragen schaut entweder auf die nächste Folie \dots
    \item \dots\ oder fragt Julian, der euch während der Zeit des eigentlichen Workshops im Learnweb
        zur Verfügung steht.
    \item Ansonsten könnt ihr auch in der Lösung im Learnweb nachschauen.
\end{itemize}
\end{frame}

\subsection{Wo finde ich Hilfe?}

\begin{frame}{\subsecname}
\begin{itemize}
    \item \LaTeX enthält bei weitem mehr Funktionalität als wir euch in diesem Workshop präsentieren
        können.
    \item Oft werdet ihr Fehler beim Kompilieren bekommen.
\end{itemize}
Deshalb ein paar Tips, wo man Hilfe und weiterführende Informationen findet:
\begin{itemize}
	\item Suchmaschine eurer Wahl mit passenden (vorzugsweise englischen) Begriffen oder Ausgaben des Compilers füttern.
	\begin{center}
		\includegraphics[keepaspectratio,width=8cm]{images/google.png}
	\end{center}
	\item LaTeX Stack Exchange: \url{https://tex.stackexchange.com}
    \item In der Dokumentation eines Pakets nachschauen, verfügbar auf \url{https://ctan.org}
\end{itemize}
\end{frame}

\section{Weiteres}

% TODO: Zitieren mit Biblatex
%\subsection{Zitieren}

\subsection{Referenzen}

\begin{frame}{\subsecname}
\begin{itemize}
    \item Wie vorher gesehen, werden z.B. Formeln mit der \monospaced{equation}-Umgebung nummeriert.
        Im restlichen Dokument kann man nun darauf \alert{referenzieren}:
        \code{references.tex}
    \item Mit \lstinline{\\label\{<name>\}} definiert man an der Stelle der Nummerierung ein
        \alert{Label}, auf das man später mit \lstinline{\\ref\{<name>\}}  verweisen kann.
        \begin{itemize}
            \item Spezialfall: \lstinline{\\eqref\{<name>\}} bei Gleichungen und Formeln
        \end{itemize}
    \item Dies funktioniert nicht nur mit Gleichungen, sondern allem, was nummeriert wird, z.B.
        Abschnitten und Figuren.
\end{itemize}
\end{frame}

\subsection{Eigene Makros}

\begin{frame}{\subsecname}
\begin{itemize}
    \item Wenn man denselben Text/dieselbe Konstruktion häufig nutzt, kann es manchmal nützlich
        sein, dafür ein neues \alert{Makro} zu erstellen:
        \code{macro-1.tex}
    \item Der Befehl \lstinline{\\newcommand\{\\makro\}[n]\{...\#1...\#2...\#n...\}}
        \begin{itemize}
            \item erstellt ein Makro namens \lstinline{\\makro}
            \item mit $n$ Argumenten
            \item auf die im sog. \textit{Body} mit \lstinline{\#1}, \lstinline{\#2}, \dots,
                \lstinline{\#n} referenziert wird.
        \end{itemize}
        Es wird aufgerufen durch \lstinline{\\makro\{arg1\}\{arg2\}...\{argn\}}
    \item Im obigen Beispiel wird \lstinline{\\iprod\{v\}\{w\}} ersetzt durch
        \begin{center}
            \lstinline{\\left\\langle v, w \\right\\langle}
        \end{center}
\end{itemize}
\end{frame}
% TODO: Makros erweitern um \DeclarePairedDelimiter und \DeclareMathOperator

% TODO: Einbindung von Source Code mit lstlistings
%\subsection{Source Code}

\end{document}
