\begin{frame}{Eigene Makros (für mathematische Formeln)}
    \begin{itemize}
        \item Mit dem Paket \texttt{mathtools} kann man kinderleicht eigene Makros für Klammern und
            Operatoren definieren.
        \item Klammern: \code{\\DeclarePairedDelimiter\{\\macro\}\{<left>\}\{<right\}}.
            \example*{examples/paired_delimiter.tex}
            \begin{itemize}
                \item Vorteil: Es ist einfach, die Größe der Klammern zu ändern (s. Dokumentation
                    von \texttt{mathtools}).
            \end{itemize}
        \item Operatoren: \code{\\DeclareMathOperator\{\\macro\}\{<name>\}}.
            \example{examples/operator.tex}
            \begin{itemize}
                \item Vorteil: Der Abstand zwischen dem Operator und dem Operanden ist korrekt.
            \end{itemize}
    \end{itemize}
\end{frame}
