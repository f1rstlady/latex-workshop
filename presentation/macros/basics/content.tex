\subsection{Eigene Makros}

\begin{frame}[fragile]{\subsecname}
    \begin{itemize}
        \item Wenn man denselben Text/dieselbe Konstruktion häufig nutzt, kann es manchmal nützlich
            sein, dafür ein neues \alert{Makro} zu erstellen:
            \example*{example.tex}
        \item Der Befehl \code{\newcommand{\macro}[n]{...#1...#2...#n...}}
            \begin{itemize}
                \item erstellt ein Makro namens \code{\macro}
                \item mit $n$ Argumenten,
                \item auf die im sog. \textit{Body} mit \code{#1}, \code{#2}, \dots, \code{#n}
                    referenziert wird.
            \end{itemize}
            Es wird aufgerufen durch \code{\macro{arg1}{arg2}...{argn}}
        \item Im obigen Beispiel wird \code{\iprod{v}{w}} ersetzt durch
            \begin{center}
                \code{\left\langle v, w \right\langle}
            \end{center}
    \end{itemize}
\end{frame}
